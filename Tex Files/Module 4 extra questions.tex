\documentclass[11pt]{article}
\usepackage[margin=0.8in]{geometry}
\usepackage{bm}
\usepackage{amsfonts}
\usepackage{amsthm}
\usepackage{amssymb}
\usepackage{amsmath}
\usepackage{xcolor}
\usepackage{tikz}
\usepackage[colorlinks = true]{hyperref}

\usepackage{fancyhdr}
\pagestyle{fancy}
\fancyhf{}
\lhead{UNSW Business School}
\rhead{ACTL3182 Module 4}
\cfoot{Page \thepage}  %PERHAPS Add UNSW COPYRIGHT FOOTER
\setlength{\headheight}{17pt} 

\title{\textbf{ACTL3182 Module 4 Extra Questions: Radon-Nikodym Theorem, Martingale and It\^o's Lemma}}
\author{Andrew Wu}
\date{October 2020}


\newcommand{\E}{\mathbb{E}}
\newcommand{\PR}{\mathbb{P}}
\newcommand{\R}{\mathbb{R}}
\newcommand{\Cov}{\operatorname{Cov}}
\newcommand{\Q}{\mathcal{Q}}

\begin{document}
	\maketitle
	\begin{enumerate}
		\item Consider two independent coin flips; that is let the sample space be $\Omega:=\{HH, HT, TH, TT\}$. Under the $\PR$ measure, the coin is fair. Under the $\Q$ measure, the coin has a probability of $\frac{1}{3}$ of landing on heads on each flip. 
		\begin{enumerate}
			\item Find the probability of each scenario under $\PR$ and $\Q$.
			\item Find the Radon-Nikodym Derivative $\frac{d\Q}{d\PR}$.
			\item Show that for the event $A$, where $A$ represents getting a head on the first flip, that
			\[	\Q(A) = \E_{P}\left[\frac{d\Q}{d\PR}1_{A}\right]
					\]
 		\end{enumerate}
		\item Consider the stochastic process 
		\[	X_n = \sum_{k=1}^{n}Z_{k},\quad X(0) = 0
			\]
		where $Z_{k}$, $k=1,2,...n$ are iid with distribution 
		\[	Z_{k} = \begin{cases}
						+1 &  \quad \text{w.p. } \frac{1}{2} \\
						-1 & \quad \text{w.p. } \frac{1}{2}.
						\end{cases}
			\]
			\begin{enumerate}
				\item What is $\E[X_{n+1}|\mathcal{F}_{n}]$?
				\item Briefly explain why $X_n$ is a martingale.
				\item Now consider a time interval $[0, T]$ and divide the interval into $n$ equal subintervals of length $\Delta = T/n$. Let $X_t^{n}$ be the process
				\[	X_t^{n} = \sqrt{\Delta}\sum_{k=1}^{\lfloor t/\Delta \rfloor}Z_{k}, \quad \text{for } 0\leq t\leq T. 
				\]
				
				Draw a possible graph of $X^{n}_{t}$ for $n=1,2$. What happens as $n\to\infty$?
				\item Calculate $\E[X^{n}_{t}]$ and $\text{Var}[X^{n}_{t}]$.
				\item Using the Central Limit Theorem, determine the limiting distribution of $X^{n}_{t}$.
				\item What continuous-time process does $X^{n}_{t}$ converge to? (no formal justification required)
			\end{enumerate}
		\item Consider the continuous-time Radon-Nikodym derivative $\zeta(t)$:
		\[	\zeta(t) = \exp\left(-\int_{0}^{t}\gamma_{s}dW(s) -\frac{1}{2} \int_{0}^{t}\gamma_{s}^{2}ds\right)
			\]
		Prove that $\zeta(t)$ is a martingale.
		\item (Challenge) Consider the stochastic differential equation (SDE)
		\[	dX(t) = \alpha(\mu - X(t))+ \sigma dW(t),\quad X(0) = x_{0}.
		\]
		The solution to this equation is called the Ornstein-Uhlenbeck process and can be used to model a massive particle moving under Brownian Motion when friction is accounted for.
		\begin{enumerate}
			\item Using It\^{o}'s Lemma, show that the solution to the SDE above 
			is 
			\[	X(t) = e^{-\alpha t}x_{0} + \mu(1 - e^{-\alpha t}) + \sigma\int_{0}^{t}e^{-\alpha(t - s)}dW(s).
			\]
			\item Explain why 
			\[	\E[X(t)] = e^{-\alpha t}x_{0} + \mu(1 - e^{-\alpha t}).
				\]
			Hint: What type of stochastic process is $\int_{0}^{t}e^{-\alpha(t - s)}dW(s)$?
			\item The It\^{o} isometry is a special property that allows the expected value of the product of It\^{o} integrals to be calculated using Riemann integrals instead. That is, for two suitable stochastic processes $X(t)$, $Y(t)$, we have:
			\[	\E\left[\int_{0}^{t}X(s)dW(s)\int_{0}^{t}Y(s)dW(s)\right]
				 = \E\left[\int_{0}^{t}X(s)Y(s)ds\right]
				\]
			Assuming this property holds for $X(t)$, show that
			\[	\text{Var}[X(t)] = \frac{\sigma^{2}}{2\alpha}(1 - e^{-2\alpha t}).
				\]
			\item What is it the distribution of $X(t)$? Explain this intuitively.
			\item Show that for any $s, t\geq 0$:
			\[	\text{Cov}(X(s), X(t)) = \frac{\sigma^{2}}{2\alpha}(e^{-\alpha|t - s|} - e^{-\alpha(s + t)})
				\]
		\end{enumerate}
	\item The price at time $t$ of a put option is given by 
	\[		p(t, S_t) = Ke^{-r(T - t)}N(-d_{2}) - S_t N(-d_1),
				\]
	where
	\[		d_{1} = \frac{\ln(S_t/ K) + (r + \frac{1}{2}\sigma^2)(T -t)}{\sigma \sqrt{T - t}},\qquad d_{2} = d_{1} - \sigma \sqrt{T - t}.
			\]
	Derive the expressions for the replicating portfolio $(\phi(t), \psi(t))$ given that
	\[		\phi(t) = \frac{\partial p}{\partial S}(t, S_t),\qquad \psi(t)=\frac{p(t, S_t) - S_t \phi(t)}{B(t)}.
			\]
	\end{enumerate}
	\textbf{Select Answers:}\\
	1. (b) The Radon-Nikodym derivative is the ratio of the two probabilities;
	\[	\frac{d\mathcal{Q}}{d\PR} = \begin{cases}
										4/9 & \quad \text{if }HH\\
										8/9 & \quad \text{if }HT\\
										8/9 & \quad \text{if }TH\\
										16/9 & \quad \text{if }TT\\
										\end{cases}
			\] 
	2. (a) $X(t)\quad$ (d) $\mathbb{E}[X^{n}_{t}] = 0$, $\text{Var}[X^{n}_{t}] = \Delta \lfloor t/ \Delta\rfloor\approx t \text{ for $n$ large}\quad$ (e) $\mathcal{N}(0,t)\quad$ (e) Brownian Motion.\\\\
	4. (d) $\mathcal{N}\left(e^{-\alpha t}x_{0} + \mu(1 - e^{-\alpha t}), \frac{\sigma^{2}}{2\alpha}(1 - e^{-2\alpha t})\right)$
	\end{document}