\documentclass[11pt]{article}
\usepackage[margin=0.8in]{geometry}
\usepackage{bm}
\usepackage{amsfonts}
\usepackage{amsthm}
\usepackage{amssymb}
\usepackage{amsmath}
\usepackage{xcolor}

\usepackage{fancyhdr}
\pagestyle{fancy}
\fancyhf{}
\lhead{UNSW Business School}
\rhead{ACTL3182 Cheatsheet}
\cfoot{Page \thepage}  %PERHAPS Add UNSW COPYRIGHT FOOTER
\setlength{\headheight}{17pt} 

\title{\textbf{ACTL3182 Formula Cheatsheet}}
\author{Andrew Wu}
\date{September 2020}

\usepackage{titlesec}
\titleformat{\section}{\Large\sffamily\bfseries\color{red}}{\thesection}{0.5em}{}[]
\titleformat{\subsection}{\color{red!75}\large\sffamily\bfseries}{\thesubsection}{0.5em}{}
\titleformat{\subsubsection}{\color{red!50}\normalsize\sffamily\bfseries}{\thesubsubsection}{0.5em}{}

\newcommand{\E}{\mathbb{E}}
\newcommand{\PR}{\mathbb{P}}
\newcommand{\R}{\mathbb{R}}
\newcommand{\Cov}{\operatorname{Cov}}

\begin{document}
	\maketitle
	\textbf{NOTE:} Theory sections are written to simplify the text for easy revision. You may need to provide more information than what is written here for exam questions.
	\section{Modern Portfolio Theory}
	\subsection{Utility Theory}
	\subsubsection{Expected Utility Theorem}
	Individual prefers \( W_1 \) over \( W_2 \) iff \( \E[u(W_1)] \geq \E[u(W_2)] \)
	\subsubsection{Utility Axioms}
	The expected utility theorem is a consequence of the following axioms:
	\begin{enumerate}
		\item \textbf{Completeness (Comparability):} For all decisions \( A,B \), either \( A \prec B \), \( A\succ B \) or \( A\sim B \)
		\item \textbf{Transitivity:} If \( A\succ B \) and \( B\succ C \), then \( A\succ C \)
		\item \textbf{Independence:} If \( A\succ B \) then for any \( C \), 
		\[	G(A,C;\alpha) \sim G(B, C; \alpha)\]
		\item \textbf{Measurability:} If \( A\succ B\succ C \), there exists a unique \( \alpha \) such that
		\[ B\sim G(A, C;\alpha)	\]
		\item \textbf{Ranking:} Suppose \( A\succ B\succ D \), \( A\succ C\succ D \), \( B\sim G(A, D;\alpha_1) \) and \( C\sim G(A, D; \alpha_2) \). Then, if \( \alpha_1\geq\alpha_2 \), then \( B\succ D \).
		\item \textbf{Certainty Equivalent:} All gambles have a price.
	\end{enumerate}
	\subsubsection{Non-Satiation}
	Individuals always prefer more wealth to less: \( u'(w) > 0 \).
	\subsubsection{Investor types}
	\begin{tabular}{lcccc}
		\hline
		\hline
		\textbf{Type} & \textbf{Wealth Preference} & \textbf{Utility Preference} & $\bm{U''}$ & \textbf{Concavity}\\
		\hline
		Risk-Averse & \( \E[W] \succ W \) & \( U(\E[W]) > E[U(W)] \) & \( U'' < 0 \) & concave\\
		\hline
		Risk-Neutral & \( \E[W] \sim W \) & \( U(\E[W]) = E[U(W)] \) & \( U'' = 0 \) & linear\\
		\hline
		Risk-Lover &\( \E[W] \prec W \) & \( U(\E[W]) < E[U(W)] \) & \( U'' > 0 \) & convex\\
		\hline
		\end{tabular}
	\subsubsection{Risk Premium}
	The amount \( \pi(W) \) that an individual will pay to give up risk:
		\[	\pi(W) = \E[W] - c(W)\]
	where \( c(W) := U^{-1}(\E[U(W)]) \) is the \textbf{certainty wealth equivalent}.
	\subsubsection{Risk Aversion}
	Absolute risk aversion \( A(w) \) and relative risk aversion \( R(w) \)
	\[	A(w) = -\frac{U''(w)}{U'(w)},\quad R(w) = -w\frac{U''(w)}{U'(w)}.\] 
	\subsection{Investment Risk Measures}
	\subsubsection{Definition}
	A function \( \mathcal{\theta}: X\to\R\) that summarises an investment's risk with a single number.
	\subsubsection{Common Examples}
	\begin{enumerate}
		\item Variance: \( \int_{-\infty}^{\infty} (x - \mu)^2 f_{X}(x)dx \)
		\item Downside-Variance: \( \int_{-\infty}^{\mu} (x - \mu)^2 f_{X}(x)dx \)
		\item Shortfall Probability: \( \PR(X \leq L ) \)
		\item Expected Shortfall: \( \int_{-\infty}^{L}(L-x)f_{X}(x)dx \)
		\item Shortfall Variance: \( \int_{-\infty}^{L}(L-x)^2  f_{X}(x)dx \)
	\end{enumerate}
	\subsubsection{Value-at-Risk:}
	The Value-at-risk \textbf{at level }\( \bm{\alpha} \) is the maximum possible loss from holding a portfolio over a \textbf{given time period} so that the \textbf{probability of larger loss is} \(\bm{1 - \alpha}  \).\\
	In general, 
	\[	\text{VaR}(\alpha) = \mu - X_{1 - \alpha},
		\]
	where \( X_{1-\alpha} \) is the \( 1 - \alpha \)-quantile of X.
	If \( X\sim\mathcal{N}(\mu, \sigma^2) \), then 
	\(	\text{VaR}(\alpha) = \sigma Z_{\alpha}
		\).
	\subsection{Portfolio Optimisation}
	\subsubsection{Two assets}
	MVP: \[	w_A = \frac{\sigma^2_B - \sigma_{AB}}{\sigma^2_A + \sigma^2_B - 2\sigma_{AB}},\quad w_B = 1 - w_A\]
	Portfolio Variance:
	\[	\sigma_P^2 = w_A^2 \sigma_A^2 + w_B^2 \sigma_B^2 + 2w_A w_B\rho_{AB}\sigma_{A}\sigma_B\]
	
	\subsubsection{N-risky assets}
	Minimise \( \displaystyle\sigma_P^2 = \bm{w}^{\top} \Sigma \bm{w} \), \( \quad \)subject to \( \bm{1}^{\top}\bm{w}= 1 \) and \( \bm{z}^{\top}\bm{w}=\mu \).\\[5pt]
	Lagrangian: \( \mathcal{L}(\bm{w}, \lambda, \gamma)  = \frac{1}{2} \bm{w}^{\top} \Sigma \bm{w} + \lambda (1 - \bm{1}^{\top}\bm{w}) + \gamma (\mu - \bm{z}^{\top}\bm{w})\).\\[5pt]
	Constants: \( A = \bm{1}^{\top}\Sigma^{-1}\bm{1},\quad B =  \bm{1}^{\top}\Sigma^{-1}\bm{z} = \bm{z}^{\top}\Sigma^{-1}\bm{1}, \quad C = \bm{z}^{\top}\Sigma^{-1}\bm{z}, \quad\Delta = AC - B^2\)\\[5pt]
	MVP: \( \displaystyle\bm{w} = \lambda \Sigma^{-1} \bm{1} + \gamma \Sigma^{-1}\bm{z}  \), where \( \lambda = \frac{C - \mu B}{\Delta}, \gamma = \frac{\mu A - B}{\Delta} \)\\[5pt]
	Portfolio Variance: \( \sigma_P^2 = \frac{A\mu^2 -2B\mu + C}{\Delta} \)\\[5pt]
	Global MVP: \( \bm{w}_g = \frac{1}{A}\Sigma^{-1}\bm{1} \).
	
	\subsubsection{N-risky assets + risk-free}
	Minimise \( \sigma_P^2 = \bm{w}^{\top} \Sigma \bm{w} \), \( \quad \)subject to \( (\bm{z} - r_{f}\bm{1})^{\top}\bm{w} = u - r_f \).\\[5pt]
	Lagrangian: \( \mathcal{L}(\bm{w}, \lambda, \gamma)  = \frac{1}{2}\bm{w}^{\top}\Sigma\bm{w} + \gamma(\mu - r_f - (\bm{z} - r_f \bm{1})^{\top}\bm{w}) \)\\[5pt]
	MVP: \( \bm{w} = \gamma\Sigma^{-1}(\bm{z} - r_f \bm{1}) \), where \( \gamma = \frac{\mu - r_f}{A r_f^2 -2B r_f + C} \)\\[5pt]
	Portfolio Variance: \( \sigma_P^2 = \frac{(\mu - r_f)^2}{A r_f^2 -2B r_f + C} \)\\[5pt]
	Tangency Portfolio: \( \bm{w}_t = \gamma_t \Sigma^{-1}(\bm{z} - r_f \bm{1}) \), where \( \gamma_t = \frac{1}{B - Ar_f} \)
	\subsubsection{Two Fund Theorem (N-risky assets)}
	All efficient portfolios are a linear combination of any two efficient portfolios. 
	\subsubsection{One Fund Theorem (N-risk + risk-free)}
	All efficient portfolios are a linear combination of the tangent portfolio and the risk-free asset.
	\section{Asset Pricing Models}
	\subsection{CAPM}
	\subsection{CAPM Assumptions}
	\begin{enumerate}
		\item Investors choose optimal portfolios based only on mean and variance of returns. Justified if returns are normally distributed returns or utility functions are quadratic.
		\item Homogeneity: Investors agree on distribution of returns and plan over a single common period.
		\item No Market Frictions (eg. taxes, short-selling restrictions, restricted information)
		\item Investors can borrow or lend at the risk-free rate.
		\item Market Portfolio consists of all publicly traded assets.
	\end{enumerate}
	\subsubsection{Capital Market Line}
	%The line representing all efficient portfolios when there are \( N \)-risky assets and a risk-free asset.
	\[	u_{e} = r_{f} + \frac{\sigma_{e}}{\sigma_{M}}(\mu_{M} - r_{f})\]
	%where \( {}_{e}, {}_{M} \) denote the efficient and Market Portfolios respectively.
	\subsubsection{Security Market Line}
	
	\[	\mu_i = r_f + \beta_i (\mu_M - r_f)\]
	\subsubsection{Risk Decomposition}
	\begin{align*}
		\sigma_i^2 & = \beta_i^2 \sigma_M^2 + \sigma_{\xi_{i}}^2 \\
		& = \text{Systematic Risk} + \text{Non-systematic Risk}
	\end{align*}	
	\subsection{Factor Models}
	\subsubsection{Single Factor Model (SFM)}
	\[	r_i = \alpha_i + \beta_i f + \epsilon_i, \]
	where \( \alpha_i, \beta_i \) are stock-specific constants, \( f \) is the factor capturing market-wide price movement and \( \epsilon_i \) is a noise term reflecting firm-specific risk.
	\subsubsection{SFM Assumptions}
	\begin{enumerate}
		\item \( (r_i, f) \) are jointly normal
		\item \( \epsilon_i\sim\mathcal{N}(0, \sigma_{\epsilon_i}^2) \)
		\item \( \Cov(\epsilon_i, f) = 0 \)
		\item \( \Cov(\epsilon_i, \epsilon_j) = 0 \) for \( i\neq j \)
	\end{enumerate}
	\subsubsection{Risk Decomposition and Covariance}
	\begin{align*}
	& \sigma_i^2 = \beta_i^2 \sigma_f^2 + \sigma_{\epsilon_{i}}^2 = \text{Systematic Risk} + \text{Non-systematic Risk}\\
	& \sigma_{i, j} = \beta_i \beta_j \sigma_f^2
	\end{align*}
	\subsubsection{Diversification}	
	\[	R_P^2 = \frac{\beta_P^2 \sigma_f^2}{\sigma_P^2} = \frac{\text{Systematic Risk}}{\text{Total Risk}}\]
	Full diversification when \( R_P^2 = 1 \).
	\subsubsection{Multi-Factor Models}
	\[	r_i = \alpha_i + \beta_{i, 1}f_1 + \beta_{i, 2}f_2 + ... + \beta_{i, K} f_K + \epsilon_i \]
	Factors \( f_i \) may include inflation, economic growth, interest rates etc.
	\subsection{APT}
	\subsubsection{Assumptions}
	Replace rigid CAPM assumptions with more relaxed assumptions:
	\begin{enumerate}
		\item No Arbitrage in the market
		\item Large universe of securities
		\item Returns follow a factor model
	\end{enumerate}
	\subsubsection{Single-Factor APT Assumptions}
	Assume returns follow the following factor model:
		\[	r_i = \alpha_i + \beta_{i}I+\epsilon_{i}\]
	Define the standardised factor \( f:=(I - \mu_{I})/\sigma_{I} \) (\( \sim\mathcal{N}(0,1) \)) and assume:
	\begin{enumerate}
		\item \( \E[\epsilon_i] = 0 \)
		\item \( \Cov(\epsilon_i, \epsilon_j) = 0\) if \( i\neq j \)
		\item \( \Cov(\epsilon_i, I) = 0\)
		\item Unlimited short-selling
	\end{enumerate}
	\subsubsection{Single-Factor APT Returns}
	\begin{align*}
		& r_i = a_i + b_i f + \epsilon_i\\[5pt]
		& \E[r_i] = a_i,\quad \sigma_i^2 = b_i^2 + \sigma_{\epsilon_i}^2,\quad\sigma_{i, j} = b_{i}b_{j}
	\end{align*}
	Under no-arbitrage, 
	\[	a_i = \lambda_0 + \lambda_1 b_i,\]
	where \( \lambda_0 = r_f \) if there is a risk-free asset.
	\subsubsection{Multi-Factor APT Returns}
	\[	r_i = a_i + b_{i,1 }f_1 + b_{i,2} f_2 + ... + b_{i, K}f_K + \epsilon_i\]
	where
	\[	\E[r_i] = a_i = \lambda_0 + \lambda_1 b_{i,1} + \lambda_2 b_{i, 2} + ... + \lambda_K b_{i, K}\]
	and \( \lambda_0 = r_f \) if there is a risk-free asset.
	\subsection{Model Fitting and Market Efficiency}
	\subsubsection{Estimating returns}
	Using historical data to estimate mean return is not feasible: over 100 years of data would be needed to get a reliable estimate.
	\subsubsection{Investment approaches}
	\begin{enumerate}
		\item Technical analysis: Analyse historical data (price and volume).
		\item Fundamental analysis: Measure intrinsic value of stocks and compare with market price.
		\item Efficient market selection: Define acceptable risk level and form broad portfolio.
	\end{enumerate}
	\subsubsection{Types of market efficiency}
	\begin{enumerate}
		\item Strong form: Market prices reflect all information including insider information.
		\item Semi-strong form: Market prices reflect all publicly available information %(fundamental and technical analysis are useless).
		\item Weak form: Market prices reflect all historical data 
		%(technical analysis is useless).
	\end{enumerate}
	\section{Discrete Time Derivative Pricing}
	\section{Continuous Time Derivative Pricing}
	\section{Term Structure Modelling and Asset-Liability Management}
	\end{document}