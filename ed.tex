Unfortunately, the notation in the solution is not good and is hence confusing you. Here, $A$ is a set of outcomes from the sample space $\Omega$.  When integrating however, you need to use the set of realisations,  $\{Y_T(\omega):\omega\in A\}$, which is the set of values $Y$ takes for the outcomes in $A$. Or in other words, $A = \{Y_T \in B\}$, and integration needs to be over the set $B$. To convert between these two sets, you use 

$$E_{P}[g(W)1_{A}] = \int_{B} g(w) f_{W}(w) dw,$$

for any function $g$ and random variable $W$ with pdf $f_W$. So your first few steps would be

$$LHS = E_{P}[Y_T 1_{\{Y_T > B\}}] = \int_{B} e^{-\gamma w - 0.5\gamma^2 T} f_{W_T}(w) dw$$

Then, by Girsanov's theorem, you know that $W^Q_t = W_t + \gamma t$ is a Q-brownian motion. Hence, you need to show that 

$$LHS = \int_{B} e^{-\gamma w - 0.5\gamma^2 T} f_{W_t}(w) dw$$$

Note: If you are still confused as to the difference between $A$ and $B$, think about this example. Consider a sample space $\Omega = {HH, HT, TH, TT}$, where $H$ means heads, $T$ means tails, i.e. all the outcomes for two coin flips. Define a random variable $X:\Omega \to [0,1]$ so that 
$$X(\omega) = \text{no. of heads flipped}.$$

So, $X(HH) = 2, X(TH) = X(HT)=1$ and $X(TT) = 0$